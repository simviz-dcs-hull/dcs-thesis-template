%% -*- Mode:LaTeX -*-

\chapter{How to \ldots}
\label{app:how-to}

\section{Citations}
\label{app:how-to:citations}

Single author:
\begin{itemize}
\item \verb!\autocite{Nicodemus:DREOS:1965}!: \autocite{Nicodemus:DREOS:1965}
\item \verb!\citetitle{Nicodemus:DREOS:1965}!: \citetitle{Nicodemus:DREOS:1965}
\item \verb!\citeauthor{Nicodemus:DREOS:1965}!: \citeauthor{Nicodemus:DREOS:1965}
\item \verb!\citeyear{Nicodemus:DREOS:1965}!: \citeyear{Nicodemus:DREOS:1965}
\end{itemize}
Multiple authors:
\begin{itemize}
\item \verb!\autocite{Sutherland:CTHSA:1974}!: \autocite{Sutherland:CTHSA:1974}
\item \verb!\citetitle{Sutherland:CTHSA:1974}!: \citetitle{Sutherland:CTHSA:1974}
\item \verb!\citeauthor{Sutherland:CTHSA:1974}!: \citeauthor{Sutherland:CTHSA:1974}
\item \verb!\citeyear{Sutherland:CTHSA:1974}!: \citeyear{Sutherland:CTHSA:1974}
\end{itemize}
\par

\section{Figures}
\label{app:how-to:figures}

\begin{figure}
  \centering
  \includegraphics[width=0.75\linewidth]{figures/empty_lscape}
  \caption[A single landscape figure]{A single landscape figure.}
  \label{fig:how-to:figures:lscape}
\end{figure}
Figure~\ref{fig:how-to:figures:lscape} contains a single image in landscape format scaled to 75\% of
the current line width.
\par

\begin{figure}
  \centering
  \subfloat[][]{%
    \includegraphics[width=0.375\linewidth]{figures/empty_portrait}
    \label{fig:how-to:figures:portait:a}
  }
  \subfloat[][]{%
    \includegraphics[width=0.375\linewidth]{figures/empty_portrait}
    \label{fig:how-to:figures:portait:b}
  }
  \caption[A single portrait figure]{A single figure containing sub-figures
    (\protect\subref{fig:how-to:figures:portait:a} and
    \protect\subref{fig:how-to:figures:portait:b})\@.}
  \label{fig:how-to:figures:portait}
\end{figure}
Figures~\ref{fig:how-to:figures:portait:a} and \ref{fig:how-to:figures:portait:b} are sub-figures
withing figure~\ref{fig:how-to:figures:portait} but can be individually referred
to. Figures~\ref{fig:how-to:figures:portait:a} and \ref{fig:how-to:figures:portait:b} are both
scaled to 37.5\% of the current line width.
\par

\section{Tables}
\label{app:how-to:tables}

\begin{table}
  \centering
  \begin{tabular}{lcr}
    \toprule
    Left   & Middle & Right\\
    \midrule
    \ldots & \ldots & \ldots \\
    \bottomrule
  \end{tabular}
  \caption{A table caption.}
  \label{tab:how-to:tables:example}
\end{table}
\ldots
\par

%%% Local Variables:
%%% mode: latex
%%% TeX-master: "../main"
%%% End:
